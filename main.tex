\documentclass[a4paper,12pt]{article}  % Document type (article, report, book, etc.)

% ----- Preamble -----
\usepackage[utf8]{inputenc}   % Encoding
\usepackage[ngerman]{babel} % Neue deutsche Rechtschreibung
\usepackage[T1]{fontenc}      % Font encoding
\usepackage{amsmath, amssymb}
\usepackage{hyperref}
\usepackage{xurl}
\usepackage[flushmargin]{footmisc} % Fußnoten ohne Einrückung
\usepackage{verbatim} %Kommentare


\usepackage{lipsum}           % Example text (for testing)

% ----- Document -----
\begin{document}
	
	\title{Vollständige Induktion, \\ ein Beweisverfahren
	}   % Title
	\author{Tom Roth}              					 % Author
	\date{2. September 2025}                    % Date (or leave empty)
	
	 \maketitle                      % Creates title page
	 \thispagestyle{empty}
	 \newpage
	\thispagestyle{empty}
	\tableofcontents                 % Auto-generates table of contents
	
	\newpage
	\pagenumbering{arabic} % Seitenzahlen beginnen bei 1
	\setcounter{page}{1}  % optional, um sicherzustellen, dass es 1 ist
	
	\section{Mathematisches Beweisen}
	\subsection*{Beweise allgemein}
	Ein Beweis in der Mathematik ist eine logisch fehlerfreie Herleitung einer Aussage aus zuvor als wahr angesehener Aussagen. Diese zuvor als wahr angesehen Aussagen werden als \emph{Axiome} bezeichnet.
	Aussagen in der Mathemaik sind Sätze, die als wahr oder falsch angesehen werden können. Der Satz \glqq Für alle ungeraden natürlichen Zahlen $m$ und $n$ ist die Summe $m + n$ eine gerade Zahl\grqq \ wäre ein gutes Beispiel für eine mathematische Aussage.
	
	\subsection*{Beweisarten}
	\subsubsection*{Direkter Beweis}
	\subsubsection*{Beweis durch Kontraposition}
	\subsubsection*{Widerspruchsbeweis}
	\subsubsection*{Induktionsbeweis}
	
	
	
	\begin{comment}
	Aussagen können mit verschieden Aussageverknüpfungen miteinander zu logischen \emph{Konjunktionen} verknüpft werden. Diese Konjunktionen haben einen eigenen Wahrheitswert, sind also wahr oder falsch. Ihr Wahrheitswert hängt dabei nur von den Wahrheitswerten der Aussagen, aus der sie bestehen, und wie diese verknüpft wurden, ab.
	Die Konjunktion \glqq $2 > 1$ und  4 ist eine Primzahl\grqq \ besteht aus den beiden Aussagen \glqq $2 > 1$\grqq \ und \glqq 4 ist eine Primzahl\grqq \, die mit der Aussagenverknüpfung \glqq und\grqq \ verknüpft sind.
	Da bei einer und-verknüpfung beide Aussage wahr sein müssen, ist die Konjunktion falsch, da die Aussage \glqq 4 ist eine Primzahl\grqq \ flasch ist.
	\footnote{Vgl. \url{https://www.logik.uni-jena.de/logikmedia/75/einfuehrung-in-das-mathematische-beweisen.pdf} (stand 2.9.25)}
	\end{comment}
	
	% \section{Einleitung}            % First section
	\section{Verwendungszweck}
	
	\section{Begriffe}
	
	\section{Aufbau des Beweises}

	\section{Beispiel}
	
	Wir wollen die Aussage beweisen:
	
	\[
	A(n): 1 + 2 + \dots + n = \frac{n(n+1)}{2}, \quad \forall n \ge 1
	\]
	
	\subsection*{Induktionsanfang (IA)}
	Für $n=1$:
	
	\[
	1 = \frac{1 \cdot (1+1)}{2} = 1
	\]
	
	Die Aussage ist also für $n=1$ wahr. \(\checkmark\)
	
	\subsection*{Induktionsschritt (IS)}
	Angenommen, die Aussage gilt für ein beliebiges $n \ge 1$, also
	
	\[
	A(n): 1 + 2 + \dots + n = \frac{n(n+1)}{2}
	\]
	
	Wir zeigen, dass daraus $A(n+1)$ folgt:
	
	\[
	\begin{aligned}
		1 + 2 + \dots + n + (n+1) &= \frac{n(n+1)}{2} + (n+1) \\
		&= \frac{n(n+1) + 2(n+1)}{2} \\
		&= \frac{(n+1)(n+2)}{2}
	\end{aligned}
	\]
	
	Damit ist die Aussage auch für $n+1$ wahr.
	
	\subsection*{Schlussfolgerung}
	Da IA und IS gezeigt wurden, gilt die Aussage für alle \\ natürlichen Zahlen $n \ge 1$.
	
	\section{Fazit}
	
	\section{Quellen}

	
\end{document}
