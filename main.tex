\documentclass[a4paper,12pt]{article}  % Document type (article, report, book, etc.)

% ----- Preamble -----
\usepackage[utf8]{inputenc}   % Encoding
\usepackage[ngerman]{babel} % Neue deutsche Rechtschreibung
\usepackage[T1]{fontenc}      % Font encoding
\usepackage{amsmath, amssymb}
\usepackage{hyperref}
\usepackage{xurl}
\usepackage[flushmargin]{footmisc} % Fußnoten ohne Einrückung
\usepackage{verbatim} %Kommentare
\usepackage{graphicx} %bilder
%\usepackage[style=footnote, sorting=none, backend=biber]{biblatex}
%\addbibresource{literatur.bib}
 \usepackage[backend=biber,style=alphabetic]{biblatex}
 \addbibresource{literatur.bib} % Name deiner .bib-Datei



\usepackage{lipsum}           % Example text (for testing)

% ----- Document -----
\begin{document}
	
	\title{Die vollständige Induktion, \\ ein Beweisverfahren. \\
	Wie funktioniert es und warum ist es so wichtig?}
	% Title
	\author{Tom Roth}              					 % Author
	\date{2. September 2025}                    % Date (or leave empty)
	
	 \maketitle                      % Creates title page
	 \thispagestyle{empty}
	 \newpage
	\thispagestyle{empty}
	\tableofcontents                 % Auto-generates table of contents
	
	\newpage
	\pagenumbering{arabic} % Seitenzahlen beginnen bei 1
	\setcounter{page}{1}  % optional, um sicherzustellen, dass es 1 ist
	
	\section{Mathematisches Beweisen}
	Ein Beweis in der Mathematik ist eine logisch fehlerfreie Herleitung einer Aussage aus zuvor als wahr angesehener Aussagen. Diese zuvor als wahr angesehen Aussagen werden als \emph{Axiome} bezeichnet.
	Aussagen in der Mathemaik sind Sätze, die als wahr oder falsch angesehen werden können. Der Satz \glqq Für alle ungeraden natürlichen Zahlen $m$ und $n$ ist die Summe $m + n$ eine gerade Zahl\grqq \ wäre ein gutes Beispiel für eine mathematische Aussage. Dabei werden zwei Aussagen mit einer Aussageverknüpfung, hier \glqq und\grqq , \ zu einer Konjunktion verknüpft, die ihren eigenen Wahrheitswert besitzt.
	Meist sollen Aussagen der Form $\forall x(P(x) \Rightarrow Q(x))$, also wenn . Hierbei sind die beiden Aussagen $P$ und $Q$ mit einer \glqq wenn, dann\grqq -Verknüpfung miteinander verknüpft. 
	Diese Art von Konjunktionen nennt man \emph{Implikationen.}
	Es soll also bewiesen werden, dass wenn $P$ gilt, $Q$ auch gilt. \\	
	Die dazugehörige Wahrheitstabelle sieht wie folgt aus:
	\[
	\begin{array}{|c|c|c|}
		\hline
		P & Q & P \Rightarrow Q \\
		\hline
		W & W & W \\
		W & F & F \\
		F & W & W \\
		F & F & W \\
		\hline
	\end{array}
	\]
	Wie in der Tabelle zu sehen ist, ist diese Aussage nur falsch, wenn $P$ wahr ist, und $Q$ falsch.
	Diese Aussage kann allerdings auch umgeformt werden. \\
	Das Kommutativ gesetz darf nicht angewendet werde. $P \Rightarrow Q \not\equiv Q \Rightarrow P$. \\
	Die Wahrheitswerte können allerdings mithilfe des \glqq nicht\grqq -Operators negiert werde. Da die Konjunktion nur flasch ist, wenn $P$ wahr und $Q$ falsch ist gilt auch: $P \Rightarrow Q \equiv \lnot Q \Rightarrow \lnot P$. \\
	Einfache Aussagen könne auch umgeformet werden. \\ So gilt auch $P \equiv \lnot P \Rightarrow (Q \ \& \ \lnot Q)$.
	Auf Grund der Vielfalt an Aussagen gibt es verschiedene Beweisverfahren, die zum Teil auf die selben Aussagen angewendet werden können, oder verschiedene Arten von Aussagen beweisen können.
	Die gängisten werden im foglenden kurz erklärt.
	\subsection{Direkter Beweis}
	Der direkte Beweis wird bei Implikationen verwendet. Dabei nimmt man $P$ als wahr an und leitet daraus $Q$ her.
	\subsection{Beweis durch Kontraposition (auch bekannt als indirekter Beweis)}
	Bei der Kontraposition macht man sich die Umformung der Implikation zu nutze. Wie vorher gezeigt gilt  $P \Rightarrow Q \equiv \lnot Q \Rightarrow \lnot P$. Man zeigt also, dass wenn $Q$ falsch ist, $P$ auch falsch ist.
	\subsection{Widerspruchsbeweis}
	Mit dem Widerspruchsbeweis beweist man einfache Aussagen. Wenn man zum Beispiel zeigen will, dass $P$ gilt, kann man durch die oben gezeigte Umformung deim Widerspruchsbeweis auch zeigen, dass seine negation einen Widersrpuch erzeugt, also flasch ist. Denn $P \equiv \lnot P \Rightarrow (Q \ \& \ \lnot Q)$ gilt. $(Q \ \& \ \lnot Q)$ ist immer falsch, denn $Q$ kann nicht gleichzeitig richtig und flasch sein. Man könnte also auch schreiben $\lnot P \Rightarrow \bot$.
	Beim Widerspruchbeweis nimmt man also das Gegenteil der Aussage an und zeigt, dass dies zu einem Widerspruch führt.
	\subsection{Induktionsbeweis}
	Der Induktionsbeweis wird verwendet, um Aussagen für alle Elemente einer bestimmten Menge zu beweisen. Bei der Menge kann es sich um verschiedene Dinge handeln. Meist will man etwas für alle natürlichen Zahlen beweisen, es kann aber jede Menge sein, die sich \emph{induktiv} definieren lässt. Wie die Induktive Mengendefinition fuinktionert wird im nächsten Abschnitt erklärt. 
	\subsubsection{Vollständige Induktion}
	Bei der vollständigen Induktion entspricht die Menge immer den natürlichen Zahlen. Man beweist also bei der vollständigen Induktion, dass eine Aussage für alle natürlichen Zahlen gilt.
	% \cite{uni-jena}
	\footnote{Vgl. \url{https://www.logik.uni-jena.de/logikmedia/75/einfuehrung-in-das-mathematische-beweisen.pdf} (stand 2.9.25)}
	
	\newpage
	
	\section{Induktive Mengendefinition}
	Beim Festlegen einer \emph{induktiven Menge} folgt man zwei Schritten.
	 \begin{enumerate}
	 	\item Basismenge: Festlegung bestimmter Basisobjekte, die in der Menge M enthalten sein sollen.
	 	\item Erzeugungsregel: Aufstellen von Regeln, mit denen sich die anderen Elemente aus den Basisobjekten ergeben 
	 	\footnote{Vgl. \url{https://www.logik.uni-jena.de/logikmedia/75/einfuehrung-in-das-mathematische-beweisen.pdf} (stand 2.9.25)}
	 	\footnote{Vgl. \url{https://rg1-teaching.mpi-inf.mpg.de/old-ag2/teaching/dsl/v22-dsl.pdf} (stand 3.9.25)} 
	 \end{enumerate}
	 Häufig will man etwas für alle \emph{natürlichen Zahlen} beweisen. Diese haben die folgenden vier Eigenschaften: \\
	 \begin{enumerate}
	 	
	 	\item \glqq  Die kleinste natürliche Zahl ist 1.
	 	\item Jede natürliche Zahl hat eine um 1 größere Zahl. Diese ist ihr Nachfolger.
	 	\item Zwischen einer natürlichen Zahl und ihrem Nachfolger liegt keine weitere natürliche Zahl.
	 	\item Es gibt keine größte natürliche Zahl. Es gibt also unendlich viele natürliche Zahlen.
	 	\grqq
	 \footnote{Vgl. \url{https://studyflix.de/mathematik/zahlenmengen-3260/natuerliche-zahlen} (stand 3.9.25)}
	 \end{enumerate}
	 Anhand der ersten beiden Eigenschaften kann man schon erkennen, dass die natürlichen Zahlen eine induktive Menge sind.
	 
	\begin{comment}
	Aussagen können mit verschieden Aussageverknüpfungen miteinander zu logischen \emph{Konjunktionen} verknüpft werden. Diese Konjunktionen haben einen eigenen Wahrheitswert, sind also wahr oder falsch. Ihr Wahrheitswert hängt dabei nur von den Wahrheitswerten der Aussagen, aus der sie bestehen, und wie diese verknüpft wurden, ab.
	Die Konjunktion \glqq $2 > 1$ und  4 ist eine Primzahl\grqq \ besteht aus den beiden Aussagen \glqq $2 > 1$\grqq \ und \glqq 4 ist eine Primzahl\grqq \, die mit der Aussagenverknüpfung \glqq und\grqq \ verknüpft sind.
	Da bei einer und-verknüpfung beide Aussage wahr sein müssen, ist die Konjunktion falsch, da die Aussage \glqq 4 ist eine Primzahl\grqq \ flasch ist.
	\footnote{Vgl. \url{https://www.logik.uni-jena.de/logikmedia/75/einfuehrung-in-das-mathematische-beweisen.pdf} (stand 2.9.25)}
	\end{comment}


	\newpage
	
	\section{Aufbau der vollständigen Induktion}
	Die vollständige Induktionsbeweis besteht aus zwei Schritten, dem \emph{Induktionsanfang} und dem \emph{Induktionsschritt}.
	Wenn man eine Aussage $A$ für alle natürlichen Zahlen beweisen möchte sehen die beiden Schritte folgendermaßen aus: \\
	\begin{itemize}
		\item \textbf{Induktionsanfang:} Man zeigt, dass die Aussage für den Startwert (meist $n = 1$) wahr ist.
		\item \textbf{Induktionsschritt:} % Man zeigt, dass $A(n) \Rightarrow A(n+1)$ wahr ist. \\ \footnote{Vgl. \url{https://www2.math.uni-wuppertal.de/~fritzsch/lectures/vorkurs/vk_k3.pdf} (stand 3.9.25)} \\
		\begin{itemize}
			\item Man nimmt an, dass $A(n)$ wahr ist, die Aussage also für ein beliebiges $n$ zutrifft. Das ist die sogenannte \\ \emph{Induktionsvorraussetzung} oder \emph{Indutkionsannahme}.
			\item Man leitet aus der Induktionsvorraussetzung her, dass $A(n + 1) $ dann auch wahr ist. Das nennt man \emph{Induktionsschluss}.
			Die Aussage $A(n) \Rightarrow A(n + 1)$ muss also wahr sein. Dafür startet man bei $A(n)$  und erweiter diese Aussage um den $n + 1$-ten Term, und kommt dann durch Umformungen auf $A(n + 1)$.
			\footnote{Vgl. \url{https://www.logik.uni-jena.de/logikmedia/75/einfuehrung-in-das-mathematische-beweisen.pdf} (stand 2.9.25)}
			\footnote{Vgl. \url{https://studyflix.de/mathematik/vollstandige-induktion-2406} (stand 15.9.25)}
		\end{itemize}
	\end{itemize}	

	\section{Beispiele}
	\subsection{Gaußsche Summenformel}
	Der Mathematiker Carl Friedrich Gauß hat die Formel
	\[
	1 + 2 + \dots + n = \sum_{k=1}^{n} k = \frac{n(n+1)}{2}, \quad \forall n  \in \mathbb{N}, n \ge 1
	\]
	zur Berechnung der Summer der ersten n Zahlen aufgestellt. \footnote{Vgl. \url{https://studyflix.de/mathematik/gaussche-summenformel-2408}, (stand 3.9.25)} \\ 
	\\
	Folgende Aussage soll nun bewiesen werden:
	\[
		A(n) : 1 + 2 + \dots + n = \frac{n(n+1)}{2}, \quad \forall n  \in \mathbb{N}, n \ge 1
	\]
	\newpage 
	\subsubsection{Induktionsanfang (IA)}
	Zeige, dass $A(1)$ gilt:
	\\
	\begin{comment}
		\[
			A(n) = \frac{n * (n + 1)}{2} 
			\\ 
			mit n = 1:
		\]
	\end{comment}
	\[
		\begin{aligned}
			A(1) &= 1 \\
			1 &= \frac{1 * (1+1)}{2} \\
			1 &= 1 \ \checkmark
		\end{aligned}
	\]
	\\
	Die Aussage $A$ ist also für $n=1$ wahr. \(\checkmark\)
	
	\subsubsection{Induktionsschritt (IS)}
	Im Induktionsschritt soll folgendes gezeigt werden:
	\[
	A(n) = \frac{n(n + 1)}{2} \Rightarrow A(n + 1) = \frac{(n + 1)(n + 2)}{2}
	\]
	Dafür wird angenommen, dass $A(n)$ wahr ist (\emph{Induktionsvoraussetzung}):
	\[
	A(n): 1 + 2 + \dots + n = \frac{n(n+1)}{2}
	\]
	Danach wird gezeigt, dass, wenn $A(n)$ wahr ist, auch $A(n + 1)$ wahr ist \\ (\emph{Induktionsschluss}). \\
	
		\[
		\begin{aligned}
			A(n) &=  \frac{n(n+1)}{2}                 && \quad| \;\text{+ $(n + 1)$} \\
			A(n) + (n + 1) &=  \frac{n(n+1)}{2} + (n + 1) && \quad| \;\text{T} \\
			1+ 2+ \dots + n + (n + 1) &=  \frac{n(n+1)}{2} + (n + 1) && \quad| \;\text{T} \\
			A(n + 1) &= \frac{n(n+1)}{2} + (n + 1)    && \quad| \;\text{mit 2 erweitern} \\
			&= \frac{n(n+1)}{2} + \frac{2(n + 1)}{2} && \quad| \;\text{zusammenfassen} \\
			&= \frac{n(n+1) + 2(n + 1)}{2}   && \quad| \;\text{$(n + 1)$ ausklammern} \\
			&= \frac{(n+1)(n + 2)}{2}        && \quad| \;\text{Q.E.D.}
		\end{aligned}
		\]
	\begin{comment}
		\[
		\begin{aligned}
			\underbrace{1 + 2 + \dots + n}_{\text{A(n)}} + (n+1) &= \frac{n(n+1)}{2} + (n+1) \\
			&= \frac{n(n+1)}{2} + \frac{2(n+1)}{2} \\
			&= \frac{(n+1)(n+2)}{2}
		\end{aligned}
		\]
	\end{comment}
		

	
	Damit ist die Aussage $A$ auch für $n+1$ wahr. \checkmark
	
	\subsubsection{Schlussfolgerung}
	Aus Induktionsanfang und Induktionsschritt folgt, dass die Aussage $A$ für alle natürlichen Zahlen $n \ge 1$wahr ist.
	
	\subsection{Beispiel aus der analytischen Geometrie: Die maximale Anzahl an Schnittpunkten von n Geraden}
	Es sollen $n$ zueinander nicht parallele so gezeichnet werden, dass sich möglichst viele Schnittpunkte $S$	ergebe.
	Skizze:
		\begin{figure}[h]
			\centering
			\includegraphics[width=0.8\textwidth]{n_geraden_skizze.png}
			\caption{Lambacher Schweizer Mathematik für Gymnasien, Kursstufe – Leistungsfach, 2016, S. 286, Abb. 1.}
			\label{fig:Geraden mit maximaler Anzahl an Schnittpunkten Skizze}
		\end{figure}
		
		
	Für die ersten Paar Geraden ergibt sich folgende Tabelle für $n$ und $S$:
		\begin{figure}[h]
			\centering
			\includegraphics[width=0.2\textwidth]{n_geraden_tabelle.png}
			\caption{Lambacher Schweizer Mathematik für Gymnasien, Kursstufe – Leistungsfach, 2016, S. 286, Fig. 1.}
			\label{fig:Geraden mit maximaler Anzahl an Schnittpunkten Tabelle}
		\end{figure}
	
	Für den Zusammenhang zwischen Anzahl der nicht parallelen Geraden $n$ und deren maximale Anzahl an Schnittpunkte $S$ soll nun folgende Abhängigkeit bewiesen werden: \\
	
		\[
			S(n) = \frac{1}{2} \cdot n \cdot (n - 1)
		\]
	Diese Aussage soll nun bewiesen werden.
	
	\subsubsection{Induktionsanfang (IA)}
	Zuerst muss die Aussage für die erste Natürliche Zahl bewiesen werden.
	Wenn man in der Tabelle nachschaut, sollten sich bei $n=1$ $0$ Schnittpunkte ergeben.
	$S(1)$ sollte also $0$ ergeben.
	\[
	\begin{aligned}
		S(1) &= 0 \\
		0 &= \frac{1}{2} \cdot 1 \cdot (1 - 1) \\
		0 &= \frac{1}{2} \cdot 1 \cdot 0 \\
		0 &= 0 && \quad| \;\text{Q.E.D.}
	\end{aligned}
	\]
	
	\subsubsection{INduktionsschritt (IS)}
	Bei jeder nächten Gerade, also der $n + 1$-ten kommen maximal $n$ Schnittpunkte dazu, da die $n + 1$-te Gerade maximal alle bisherigen einmal schneiden kann.
	Das wird auch bei folgendem Bild klar:
	\begin{figure}[h]
		\centering
		\includegraphics[width=0.8\textwidth]{n+1_graphen_skizze.png}
		\caption{Lambacher Schweizer Mathematik für Gymnasien, Kursstufe – Leistungsfach, 2016, S. 286, Abb. 3.}
		\label{fig:Neue SChnittpunkte bei n-ter Gerade}
	\end{figure}
	\newpage
	Man nimmt also an, dass $S(n)$ wahr ist (Induktionsannahme) und zeigt, dass dann auch $S(n + 1)$ wahr ist.
	Das folgende soll also gelten:
	\[
		S(n + 1) = S(n) + n
	\]

	Das wird durch Umformen gezeigt:
	
	\[
		\begin{aligned}
			\frac{1}{2} \cdot (n + 1) \cdot ((n + 1) - 1) &= \frac{1}{2} \cdot n \cdot (n - 1) + n && \quad| \;\text{T} \\
			\frac{1}{2} \cdot (n + 1) \cdot n &= \frac{1}{2} \cdot (n^2 - n) + n  && \quad| \;\text{Ausklammern} \\
			\frac{1}{2} \cdot (n^2 + n) &= \frac{1}{2} \cdot (n^2 - n + 2n) && \quad| \;\text{T} \\
			\frac{1}{2} \cdot (n^2 + n) &= \frac{1}{2} \cdot (n^2 + n) && \quad| \;\text{Q.E.D.} 
		\end{aligned}
	\]
	
	\subsubsection{Kettneschluss}
	Da die Aussage für die erste natürliche Zahl gilt(IA) und für alle Nachfolger(IS), gilt die Aussage für alle natürlichen Zahlen.
	\footnote{Vgl. Lambacher Schweizer Mathematik für Gymnasien, Kursstufe – Leistungsfach, 2016, S. 286, Problem 1}

	\section{Fazit}
		Die vollständige Induktion ist ein Beweiseverfahren, dass sich eignet, um eine Aussage für alle natürliche Zahlen zu beweisen. 	
		Es ist sehr effizient, da man in nur drei Schritten unendlich viele Fälle beweist. So müsste man bei den beiden Beispielen ohne die vollstände Induktion die Aussagen für alle natürliche Zahlen einzeln beweisen.
		\footnote{Vgl. \url{https://studyflix.de/mathematik/vollstandige-induktion-2406} (stand 18.11.25)}
 \newpage
	\section{Quellen}
	% \printbibliography
	% \bibliographystyle{plain}
	% \bibliography{literatur.bib}
	\subsection{Textquellen}
	\begin{enumerate}
		\item{https://www.logik.uni-jena.de/logikmedia/75/einfuehrung-in-das-mathematische-beweisen.pdf (stand 2.9.25)}
		\item{https://rg1-teaching.mpi-inf.mpg.de/old-ag2/teaching/dsl/v22-dsl.pdf (stand 3.9.25)}
		\item{https://studyflix.de/mathematik/zahlenmengen-3260/natuerliche-zahlen (stand 3.9.25)}
		\item{https://studyflix.de/mathematik/vollstandige-induktion-2406 (stand 15.9.25)}
		\item{https://studyflix.de/mathematik/gaussche-summenformel-2408 (stand 3.9.25)}
		\item{Lambacher Schweizer Mathematik für Gymnasien, Kursstufe – Leistungsfach, 2016, S. 286, Problem 1}
		
		
	\end{enumerate}
	\subsection{Bildquellen}
	\begin{enumerate}
		\item{Lambacher Schweizer Mathematik für Gymnasien, Kursstufe – Leistungsfach, 2016, S. 286, Abb. 1.}
		\item{Lambacher Schweizer Mathematik für Gymnasien, Kursstufe – Leistungsfach, 2016, S. 286, Fig. 1.}
		\item{Lambacher Schweizer Mathematik für Gymnasien, Kursstufe – Leistungsfach, 2016, S. 286, Abb. 3.}
	\end{enumerate}
\end{document}


