\documentclass[a4paper,12pt]{article}  % Document type (article, report, book, etc.)

% ----- Preamble -----
\usepackage[utf8]{inputenc}   % Encoding
\usepackage[ngerman]{babel} % Neue deutsche Rechtschreibung
\usepackage[T1]{fontenc}      % Font encoding
\usepackage{amsmath, amssymb}

\usepackage{lipsum}           % Example text (for testing)

% ----- Document -----
\begin{document}
	
	\title{Vollständige Induktion
			ein Beweisverfahren
	}   % Title
	\author{Tom Roth}               % Author
	\date{2. September 2025}                    % Date (or leave empty)
	
	 \maketitle                      % Creates title page
	 \thispagestyle{empty}
	 \newpage
	\thispagestyle{empty}
	\tableofcontents                 % Auto-generates table of contents
	
	\newpage
	\pagenumbering{arabic} % Seitenzahlen beginnen bei 1
	\setcounter{page}{1}  % optional, um sicherzustellen, dass es 1 ist
	
	\section{Einleitung}            % First section
	
	\section{Zweck}
	
	\section{Verfahren}

	\section{Beispiel}
	
	Wir wollen die Aussage beweisen:
	
	\[
	A(n): 1 + 2 + \dots + n = \frac{n(n+1)}{2}, \quad \forall n \ge 1
	\]
	
	\subsection*{Induktionsanfang (IA)}
	Für $n=1$:
	
	\[
	1 = \frac{1 \cdot (1+1)}{2} = 1
	\]
	
	Die Aussage ist also für $n=1$ wahr. \(\checkmark\)
	
	\subsection*{Induktionsschritt (IS)}
	Angenommen, die Aussage gilt für ein beliebiges $n \ge 1$, also
	
	\[
	A(n): 1 + 2 + \dots + n = \frac{n(n+1)}{2}
	\]
	
	Wir zeigen, dass daraus $A(n+1)$ folgt:
	
	\[
	\begin{aligned}
		1 + 2 + \dots + n + (n+1) &= \frac{n(n+1)}{2} + (n+1) \\
		&= \frac{n(n+1) + 2(n+1)}{2} \\
		&= \frac{(n+1)(n+2)}{2}
	\end{aligned}
	\]
	
	Damit ist die Aussage auch für $n+1$ wahr.
	
	\subsection*{Schlussfolgerung}
	Da IA und IS gezeigt wurden, gilt die Aussage für alle \\ natürlichen Zahlen $n \ge 1$.
	
	\section{Fazit}

	
\end{document}
